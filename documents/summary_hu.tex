\documentclass[12pt]{article}
\usepackage[utf8]{inputenc}
\usepackage{t1enc}
\def\magyarOptions{defaults=hu-min}
\usepackage[magyar]{babel}
\title{\vspace{-2.0cm}Összefoglaló}
\begin{document}
    \date{\vspace{-5ex}}
    \maketitle
    A neurális hálók és a mélytanulás népszerűsége, valamint jelentősége folyamatosan növekedett az elmúlt évtizedek óta.
    Ezek a modellek egyre inkább átveszik a vezető szerepet a gépi tanulás legnehezebb problémáinak megoldása során,
    folyamatosan kiválóbb és ígéretesebb eredményeket produkálva.\\
    \indent Az egyik nagy területe az alkalmazott gépi tanulásnak az idősor elemzés, valamint előrejelzés, ahol a klasszikus
    gépi tanulásos és mélytanulásos módszereket is használnak manapság a legjobb eredményekért.
    Sok kategóriába sorolhatóak a területnek a különféle kérdései és adattípusai. Ma az egyik legtöbb kihívást
    jelentő feladat a pénzügyi piac árainak előrejelzése jelenti több okból is kifolyólag. A jel-zaj viszony jellemzően alacsony,
    különösen nagy mintavételezési frekvenciával és nagyon kitett a külső politikai vagy gazdasági tényezőknek, amiket nem lehet
    előre látni. Emellett olyan mintákat tud mutatni, amik időben nem állandóak -
    , ha valaki talál egy prediktív mintázatot és elkezdi kihasználni, az magával vonja a minta változását is. A felsorolt
    okokból is kifolyólag az adat előkészítése a tanításra sok nehézséget okoz.\\
    \indent A Legújabb mélytanulásos architektúrák gyakran hordoznak magukkal valamilyen figyelem mechanikát, ami
    azt teszi lehetővé a hálónak, hogy a kimenet különböző részeinek létrehozásakor a bemenet egyes részeire jobban
    fókuszáljon. Ez sokat segít például képek felcímkézésénél (csak az objektum környezetére figyelünk, annak osztályának eldöntésénél)
    vagy esetleg fordításnál, ahol egyes fordított szavak jobban függnek a bemenet valamely szavaitól.
    Manapság az egyik legkorszerűbb képgenerálás vagy fordítási problémák megoldásának gerince az
    úgynevezett Transformer modell, ami ezt a figyelmi mechanizmust használja rendkívül effektíven.\\
    \indent Dolgozatomban a Transformer alapú modellek pontosságát vizsgálom idősor osztályozásra. Munkámban a Bitcoin
    árak közeljövőbeli tendenciáit próbálom előrejelezni Bitcoin és közösségi média adatokból, mivel a kriptovaluták
    általánosságban még mindig ki vannak téve felhajtásnak és trendeknek. Munkám során néhány népszerű, klasszikus és
    mélytanulásos technikát is megvizsgálok, annak érdekében, hogy egy viszonyítási alapot kapjak a Transformer-inspirálta
    modell teljesítményéhez.
\end{document}
